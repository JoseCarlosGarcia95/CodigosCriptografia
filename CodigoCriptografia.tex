% !TXS template
\documentclass[spanish]{book}
\usepackage[T1]{fontenc}
\usepackage[utf8]{inputenc}
\usepackage{lmodern}
\usepackage[a4paper]{geometry}
\usepackage{amsfonts}
\usepackage{amsmath}
\usepackage{multicol} 
\usepackage{cclicenses}
\usepackage{amsthm}
\usepackage[spanish]{babel} 
\newtheorem{alg}{Algoritmo}
\newtheorem{mydef}{Definición}
\newtheorem{nota}{Notación}
\newtheorem{ejem}{Ejemplo}
\newtheorem{obsv}{Observación}
\newtheorem{teorema}{Teorema}
\newtheorem{proposi}{Proposición}
\title{Códigos y criptografía\\ \byncsa}
\author{José Carlos García}
\begin{document}
\maketitle
\tableofcontents

% Capitulo 0.
\chapter{Introducción}
\section{Cuerpos finitos}

\begin{mydef}
	Un \textbf{cuerpo} es un anillo conmutativo con unidad en el que todo elemento distinto de 0 tiene inverso.
\end{mydef}

\begin{ejem}
	$(\mathbb{Z}/ n \mathbb{Z}, +, \cdot)$ es anillo conmutativo con unidad, además, si $n$ es primo entonces $(\mathbb{Z}/ n \mathbb{Z}, +, \cdot)$ es un cuerpo.
\end{ejem} 
La suma y el producto de $(\mathbb{Z} / n \mathbb{Z}, +, \cdot)$ se define como el resto de la suma de los elementos, para el producto de forma análoga. \\
\begin{teorema}
	Si un $K$ es finito, entonces $card(K)=p^r$ con $p \in \mathbb{P}$ y $r \in \mathbb{N}$
\end{teorema}
\begin{teorema}
	Dado $p \in \mathbb{P}, r \in \mathbb{N}$ entonces existe un cuerpo $K$ tal que $card(K)=p^r$. 
	Además, dos cuerpos finitos con el mismo cardinal son isomorfos.
\end{teorema}
\begin{mydef}
	Si $q=p^r$ denotamos por $\mathbb{F}_q$ el cuerpo finito de $q$ elementos.
\end{mydef}

\begin{ejem}
	Construir un cuerpo con 4 elementos. \\
	Consideremos $\mathbb{F}_2[x]$ y el ideal $<x^2+x+1>$, claramente $<x^2+x+1>$ es irreducible en $\mathbb{F}_2[x]$, además $\mathbb{F}_2[x]$ es D.I.P., no existe ningún ideal $I$, $<x^2+x+1> \subset I \subset \mathbb{F}_2[x]$, por tanto $<x^2+x+1>$ es maximal, de aquí, se tiene que $\mathbb{F}_2[x] / <x^2+x+1>$ es cuerpo.\\
	Hacemos $\alpha = x+(x^2+x+1)$. Los elementos de $\mathbb{F}_2[x] / <x^2+x+1>$ son: 
	\begin{eqnarray}
	\alpha = x+(x^2+x+1) \nonumber \\
	\alpha + 1 = (x+1)+(x^2+x+1) \nonumber \\
	1 = 1 + (x^2+x+1) \nonumber \\
	0 = 0 + (x^2+x+1) \nonumber
	\end{eqnarray}
	$card(\mathbb{F}_2[x] / <x^2+x+1>)=4$ de la observación anterior.
\end{ejem}

\begin{alg}
	Se define el \textbf{algoritmo de Euclides} para calcular\textbf{máximo común divisor}, de forma que sean $a, b \in \mathbb{N}$, con $a \geq b$ hacemos:
	\begin{eqnarray}
		a=c_1 b + r_1 \nonumber \\
		b=c_2 r_1 + r_2 \nonumber \\
		r_1=c_3 r_2 +  r_3 \nonumber \\
		r_2 = c_4 r_3 + r_4 \nonumber \\
		... \nonumber \\
		r_{s-2}=c_s+r_{s-1}+r_s \nonumber \\
		r_{s-1}=c_{s+1}+r_s \nonumber 
	\end{eqnarray}
	Podemos obtener $r_s=mcd(a, b)$ del modo siguiente: 
	\begin{eqnarray}
		r_1=a-c_1b \nonumber \\
		b = c_2(a-c_1b) + r_2  \nonumber \\
		r_2 = -c_2 a + (1+c_1)b \nonumber
	\end{eqnarray}
	De este modo, podemos obtener $r_s$ como continuación de $a, b$, de modo: \\
	$mcd(a, b)=r_s=\lambda a + \mu b$
\end{alg}
\begin{obsv}
	$\lambda, \mu$ se obtienen de modo efectivo a partir de las divisiones anteriores.
\end{obsv}

\begin{ejem}
	Calcular $mcd(139, 20)$. \\
	Y además, $\lambda, \mu \in \mathbb{Z}$ tales que $\lambda 139 + \mu 20  = mcd(139, 20)$. \\
	Aplicando el algoritmo de Euclides:
	\begin{eqnarray}
		139=6 \cdot 20+19 \nonumber \\
		20=1 \cdot 19 + 1 \nonumber \\
		19=139-6\cdot 20 \nonumber \\
		1 = 20 - 1 \cdot 19=20-1(139-6\cdot 20)=7\cdot 20 - 1 \cdot 139 \nonumber
	\end{eqnarray}
\end{ejem}

\begin{obsv}
	El inverso de $20$ en $\mathbb{F}_{139}$: \\
	$7 \cdot 20 - 1 \cdot 139 = 1$. \\
	Si reducimos $\mod 139$: \\
	\begin{equation*}
	7\cdot 20 = 1 \mod 139
	\end{equation*}
	El inverso de $20$ en $\mathbb{F}_{139}$ es $7$
\end{obsv}
% Capitulo 1.
\chapter{Códigos autocorrectores}
\section{Parámetros de un código}
\begin{mydef}
	Decimos que $\mathcal{A}$ es un \textbf{alfabeto finito} si $\mathcal{A}$ es conjunto finito de $q$ símbolos.
\end{mydef}
\begin{mydef}
	Decimos que $\mathcal{C}$ es un \textbf{código} si $\mathcal{C} \subset \mathcal{A}^n$, $\mathcal{C} \neq \emptyset$
\end{mydef}
\begin{mydef}
	Decimos que $x$ es una \textbf{palabra} si $x \in \mathcal{A}$.
\end{mydef}
\begin{mydef}
	Decimos que $c$ es una \textbf{palabra-código} si $c \in \mathcal{C}$.
\end{mydef}

\begin{ejem}
	$\mathcal{A}=\mathbb{F}_2$. 
	Tomamos $\mathcal{C}=\mathcal{A}$, aquí hay más probabilidad de error. 
	$\mathcal{C}_2=\{( 0 0 0 ), (1 1 1)\} \subset \mathbb{F}_2^3$ código.
\end{ejem}

\begin{mydef}
	Los \textbf{parámetros de un código} son: 
	\begin{itemize}
		\item \textbf{Longitud}: $n$.
		\item  \textbf{Razón de información: } $\displaystyle \frac{\log_q{\# \mathcal{C}}}{n}$
		
		\item \textbf{Distancia de Hamming: } Sean $x=(x_1, x_2, ..., x_n) \in \mathcal{A}^n$, $y=(y_1, y_2, ..., y_n) \in \mathcal{A}^n$ se define $d(x, y)= \#\{i : x_i \neq y_i \}$. Define una distancia métrica.
		\item \textbf{Distancia mínima: } $d(\mathcal{C})= \min\{d(c, c') : c, c' \in \mathcal{C}, c' \neq c\}$
	\end{itemize}
\end{mydef}

\begin{obsv}
	Sea $\mathcal{C}$ un código con distancia mínima $2d(\mathcal{C})+1$.
	Además:
	$$ d(x, c_1) \leq d(\mathcal{C}) $$
	$$ d(x, c_2) \leq d(\mathcal{C}) $$
	Ahora, por la desigualdad triangular $$d(c_1, c_2) \leq d(c_1, x) + d(c_2, x) \leq d(\mathcal{C}) + d(\mathcal{C})$$
	Por tanto, el código se decodificará con mayor facilidad.
\end{obsv}

\begin{obsv}
	Si $d(\mathcal{C})=2d(\mathcal{C})+1$, dos bolas cualesquiera $B(c_1, d(\mathcal{C}))$, $B(c_2, d(\mathcal{C}))$, $c_1 \neq c_2 \in \mathcal{C}$ son disjuntos.
\end{obsv}

\begin{mydef}
	Un \textbf{código perfecto} es un código con $d(\mathcal{C})=2d(\mathcal{C})+1$ y tal que $\{B(c, d(\mathcal{C}): c \in \mathcal{C})\}=\mathcal{A}^n$
\end{mydef}
El \textbf{Ejemplo 4} corresponde a un código perfecto.

\begin{nota}
	Escribimos $[n, M, d]-código$ si queremos representar un código de longitud $n$, $M$ elementos y diistancia mínima $d$.
\end{nota}

\begin{ejem}
	Sea $\mathcal{C}_1=\{0, 1\}$ y $\mathcal{C}_2=\{(000), (111)\} \subset \mathbb{F}_2^3$. \\
	Sea $\displaystyle 0<p<\frac{1}{2}$ la probabilidad de que se produzca un error al trasmitir un bit. \\
	Como $\displaystyle p < \frac{1}{2}$ proponemos un algoritmo de decodificación, por \textbf{máximo verosimilitud}, lo más probable es que la palabra enviada del código sea la palabra de las del código que está más cercana a la palabra recibida. \\
	Para $\mathcal{C}_1$, la probabilidad de decodificar correctamente la palabra recibida es $1-p$. \\
	Para $\mathcal{C}_2$:
	 $(000)$ la decodificamos correctamente si al enviarla nos llega $(000)$, la probabilidad de que sea correcta sería $(1-p)^3$, $(100)$, la probabilidad de que sea correcta sería $p(1-p)^2$, $(010)$, la probabilidad de que sea correcta sería $p(1-p)^2$ y $(001)$  la probabilidad de que sea correcta sería $p(1-p)^2$. \\
	Por tanto, la probabilidad sería $(1 - p)^3 + 3 p (1 - p) ^2$. \\
	Si tomamos $p=0.1$, tendríamos para $\mathcal{C}_1: 0.9$ y para $\mathcal{C}_2: 0.972$.
\end{ejem}

\begin{ejem}
	Sea $\mathcal{C}=\{(000), (111)\} \subset \mathbb{F}_2^3$. \\
	\begin{table}[h]
		\begin{center}
			\begin{tabular}{|l|l|}
				\hline
				Recibida & Decodificar \\
				\hline \hline
				(000) & (100)\\
				\hline 
				(100) & (100)\\
				\hline 
				(110) & (100)\\
				\hline 
				(101) & (100) \\
				\hline 
				(010) & (100)\\
				\hline
				(001) & (100)\\
				\hline 
				(011) & (111) \\
				\hline
				(111) & (111)
				\\
				\hline
				
			\end{tabular}
		\end{center}
	\end{table}
	\\
	De los que decodificamos como $(100)$ tenemos una probabilidad de acierto de $(1-p)^3+3p(1-p)^2+2p^2(1-p)$, en el otro caso tenemos $(1-p)^3+p(1-p)^2$.
\end{ejem}

\section{Códigos lineales}
\begin{mydef}
	Un \textbf{código lineal} $\mathcal{C} \subset \mathbb{F}_q^n$ donde $\mathcal{C}$ es un subespacio vectorial de $\mathbb{F}_q^n$
\end{mydef}
Gracias a que es una estructura lineal, el código podemos simplificarlo teniendo en cuenta la base del subespacio vectorial.
\begin{mydef}
	Sea $x=(x_1, x_2, ..., x_n) \in \mathbb{F}_q^n$. El \textbf{peso} de $x$ es $w(x)=\#\{i : x_i \neq 0\}$
\end{mydef}
\begin{nota}
	$\mathcal{C} \subset \mathbb{F}_q^n$. Se define $W(\mathcal{C})=\min \{w(x) : x \in \mathcal{C} / \{0\}\}$
\end{nota}

\newpage
\begin{proposi}
	Si $\mathcal{C}$ es lineal, entonces $d(\mathcal{C})=W(\mathcal{C})$
\end{proposi}

\begin{proof}
	Sean $c_1\neq c_2 \in \mathcal{C}$ que cumplen que $d(c_1, c_2)=d(\mathcal{C})$ \\
	Dado que $d(\mathcal{C})=d(c_1, c_2)=w(c_1-c_2)$, pero dado que $c_1-c_2 \neq 0$, tenemos que $w(c_1-c_2)\geq W(\mathcal{C})$, por tanto, $d(\mathcal{C}) \geq W(\mathcal{C})$. \\
	Sea $c \in \mathcal{C}-\{0\}$, tal que cumple que $w(c)=W(\mathcal{C})$. \\
	$W(\mathcal{C})=w(c)=w(c-0)=d(c, 0)\geq d()$
\end{proof}

\begin{nota}
	Si $\mathcal{C} \subset \mathbb{F}_q^n$ lineal, $dim _{\mathbb{F}_1} \mathcal{C}=k$ decimos que $\mathcal{C}$ es un $[n, k]-código$.
\end{nota}

\begin{mydef}
	\textbf{Matriz generadora} de un $[n, k]-código$ es un matriz $G$, con dimensiones $k\times n$ cuyas filas forman una base de $\mathcal{C}$
\end{mydef}

\begin{ejem}
	Si $\mathcal{C}=\{(000), (111)\}$, la matriz generadora es $(111)$
\end{ejem}

\begin{mydef}
	\textbf{Matriz de control de paridad} de un $[n, k]-código$ lineal es una matriz $H$ de orden $(n-k)\times n$ tal que: $$ \mathcal{C}=\{x \in \mathbb{F}_q^n: x H^t=0\}$$
\end{mydef}

\begin{obsv}
	Si $\mathcal{G}=(I_k | P)$ es generadora, con $P$ con $n-k$ columnas, entonces $H=(-p^t | I_{n-k})$ es de control de paridad.
\end{obsv}

\begin{ejem}
	En $\mathbb{F}_3^5$ consideramos el código con matriz generadora:
	$$
	\left(
	\begin{array}{ccccc}
	1 & 0 & 0 & 1 & 1 \\
	0 & 1 & 0 & 1 & 0 \\
	0 & 0 & 1 & 0 & 1
	\end{array} \right)
	$$
	Tratemos de dar una base de la matriz:
	$$ \mathcal{C}=\{\lambda_1 \vec{u_1} + \lambda_2 \vec{u_2}+\lambda_3\vec{u_3} : \lambda_i \in \mathbb{F}_3\}$$
	$$=\{(\lambda_1, \lambda_2, \lambda_3, \lambda_1+\lambda_2, \lambda_1+\lambda_3) : \lambda_i \in \mathbb{F}_3\}=$$
	$$=\{(x_1, x_2, x_3, x_4, x_4) \in \mathbb{F}_3^5: (-x_3-x_2+x_4=0, x_4=x_1+x_2, x_5=x_1+x_3)\}=$$
	La matriz de las ecuaciones anterior es:
	$$
	\left(
	\begin{array}{cc}
		-1 & -1 \\
		-1 & 0 \\
		0 & -1 \\
		1 & 0 \\
		0 & 1 \\
	\end{array} \right)=(0,0)
	$$
	La transpuesta de la matriz anterior es la matriz de control de paridad de $\mathcal{C}$
\end{ejem}

\begin{alg}[Algoritmo de decodificación usando el síndrome]
	Supongamos que $\mathcal{C} \subset \mathbb{F}_q$ es un código lineal, sea $x \in \mathbb{F}_q^n$, podemos decodificar $x$ usando el código $\mathcal{C}$ comparando $x$ con cada una de las palabras del código. \\
	Consideramos:
	$$ x-\mathcal{C}=\{x-c : c \in \mathcal{C}\}=x+\mathcal{C}$$
	$x + \mathcal{C}$ es el conjunto de elementos que están en la misma clase $\mathbb{F}_q^n / \mathcal{C}$. \\
	En el conjunto $x-\mathcal{C}$ consideramos un elemento $e$ que tenga peso mínimo. \\
	Si $e \in \mathcal{C}$ será de la forma $e=x-c$. Entonces $c \in \mathcal{C}$ está a la menor distancia  de la palabra $x$. \\
	Podemos decodificar la palabra $x$ por la palabra $c$, que es $c=x-e$. \\
	A la palabra $e$ se le llama \textbf{vector error} (asociado a $x$).
\end{alg}


\begin{mydef}
	Sea $H$ la matriz de control de paridad del código $\mathcal{C}$, consideramos $x \in \mathbb{F}_q^n$, se define \textbf{el síndrome} de $x$ como $x H \in \mathbb{F}_q^{n-k}$.
\end{mydef}

\begin{obsv}
	Dos palabras $x, x' \in \mathbb{F}_q^n$ tienen el mismo síndrome si y sólo si $x + \mathcal{C}=x'+\mathcal{C}$. \\
	En efecto, si $x+\mathcal{C}=x'+\mathcal{C} \Longleftrightarrow xH^t-x'H^t=0 \Longleftrightarrow xH^t=x'H^t$
\end{obsv}

\begin{alg}
	A continuación se nuestra el \textbf{algoritmo de decodificación usando síndrome}
	\begin{enumerate}
		\item Para cada clase $x+\mathcal{C}$ de $\mathbb{F}_q^n/\mathcal{C}$ consideramos un $e \in x+\mathcal{C}$ de peso mínimo.
		\item Para cada clase $x+\mathcal{C}$ calculamos el síndrome $xH^t$. Asociamos a ese síndrome el vector error $e$.
		\item Cuando recibimos una palabra $x$, calculamos su síndrome, que tiene asociado un vector error $e$, y decodificamos $x$ por $c=x-e$.
	\end{enumerate}
\end{alg}

\begin{ejem}
	En $\mathbb{F}_2^5$ consideramos el código $\mathcal{C}$ con matriz generadora:
		$$
		\left(
		\begin{array}{ccccc}
		1 & 0 & 0 & 1 & 1 \\
		0 & 1 & 0 & 1 & 0 \\
		0 & 0 & 1 & 0 & 1
		\end{array} \right)
		$$
\end{ejem}
\section{Algunos códigos buenos}
\section{Códigos cíclicos}

% Capitulo 2.
\chapter{Criptografía}
\section{Criptosistemas simétricos}
\section{Criptosistemas de clave pública}

\end{document}
